% SVN Info:
% $Date: 2021-02-01 14:11:20 +0100 (Mo, 01 Feb 2021) $
% $Rev: 1041 $
% $Author: kolja $
\subsection{Output Structure crono\tu packet}

	\cronvar{unsigned char}{channel}\\
	Unused, always 0.\par

	\cronvar{unsigned char}{card}\\
	Identifies the source card in case there are multiple boards present. Defaults to 0 if no value is assigned to the parameter \textsf{board\tu id} in Structure \textsf{ndigo\tu init\tu parameters}.\par

	\cronvar{unsigned char}{type}\\
	The data stream consists of 32 bit unsigned data as signified by a value of 6.\par

	\cronvar{unsigned char}{flags}\\
	\indent\crondef{TIMETAGGER4\tu PACKET\tu FLAG\tu ODD\tu HITS} 1\\
	\indent The last data word in the data array consists of one timestamp only which is located in the lower 32 bits of the 64 bit data word (little endian).\par
	\indent\crondef{TIMETAGGER4\tu PACKET\tu FLAG\tu SLOW\tu SYNC} 2\\
	\indent Start pulse distance is larger than the extended timestamp counter range.\par
	\indent\crondef{TIMETAGGER4\tu PACKET\tu FLAG\tu START\tu MISSED} 4\\
	\indent The trigger unit has discarded packets due to a full FIFO.\par
	\indent\crondef{TIMETAGGER4\tu PACKET\tu FLAG\tu SHORTENED} 8\\
	\indent The trigger unit has shortend the current packet due to full FIFO.\par
	\indent\crondef{TIMETAGGER4\tu PACKET\tu FLAG\tu DMA\tu FIFO\tu FULL} 16\\
	\indent The internal DMA FIFO was full. Might or might not result in dropped packets.\par
	\indent\crondef{TIMETAGGER4\tu PACKET\tu FLAG\tu HOST\tu BUFFER\tu FULL} 32\\
	\indent The host buffer was full. Might or might not result in dropped packets.\par

	\cronvar{unsigned int}{length}\\
	Number of 64-bit elements (each containing up to 2 TDC hits) in the data array.\par

	\cronvar{unsigned \tu\tu int64}{timestamp}\\
	Coarse timestamp of the start pulse. Values are given in multiples of $500$~ps.\par

	\cronvar{unsigned \tu\tu int64}{data[1]}\\
	TDC hits. the user can cast the array to uint32* to directly operate on the TDC hits.

	\noindent
	\begin{small}
	\begin{tabular}{|c||p{9cm}|p{1,5cm}|p{1,5cm}|}
		\hline
		bits & 31~ ~ ~ ~ ~ ~ ~ ~ ~ ~ ~ ~ ~ ~ ~ ~ ~ to ~ ~ ~ ~ ~ ~ ~ ~ ~ ~ ~ ~ ~ ~ ~ ~ ~ 8 & 7~ ~ to ~ ~ 4 & 3~ ~ to ~ ~ 0\\\hline
		content & TDC DATA & FLAGS & CHN \\\hline
	\end{tabular}
	\end{small}

	The timestamp of the hit is stored in bits 31 down to 8 in multiples of 500~ps. For the -1G Variant bit 8 is always 0.\\
	
	Bits 7 down to 4 are hit flags:\par
	Bit 7: Always 0.\par
	Bit 6: Always 1.\par
	Bit 5: Rollover. The time since start pulse exceeded the 24 range that can be enoded in a data word. This word does not ncode a hit. 
	Instead it the readout software should increment a rollover counter that can be used as upper bits of the following time stamps.  The counters should be reset for each packet.
	The total offset of a hit in picosencods can be computed by
	\[	\Delta T_{hit} = \mathrm{(\#\ rollovers \cdot 2^{23} + TDC\_ DATA_{hit}) \cdot timetagger4\_param\_info.binsize} \]
	\indent
	Bit 4: Set if this word encodes a rising edge. Otherwise this word belongs to a falling edge.
	The channel number is given in the lowest nibble of the data word. A value of 0 corresponds to channel A, a value of 3 to channel D.\par
