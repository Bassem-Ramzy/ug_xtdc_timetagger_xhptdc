% SVN Info:
% $Date: 2021-01-31 20:06:56 +0100 (So, 31 Jan 2021) $
% $Rev: 1040 $
% $Author: kolja $

	The following C++ source code shows how to initializes a \deviceName\ board, configure it and loop over incoming packets.	

	If you are reading this documentation in portable document format, the source code of the C example is also embedded as an
	\txh{
		\textattachfile[color=cronlightgreen, description={Example Source Code}]{"timetagger/example.cpp"}{attachment}
		to the file. You can open it in an external viewer or save it to disk by clicking on it.
		\lstinputlisting{timetagger/example.cpp}
	}{
		\textattachfile[color=cronlightgreen, description={Example Source Code}]{xtdc/example.cpp}{attachment}
		to the file. You can open it in an external viewer or save it to disk by clicking on it.
		\lstinputlisting{xtdc/example.cpp}
	}{	
		\textattachfile[color=cronlightgreen, description={Example Source Code}]{xhptdc/example.cpp}{attachment}
		to the file. You can open it in an external viewer or save it to disk by clicking on it.

		The example code is managed as open source on GitHub at 
		\href{https://github.com/cronologic-de/xhptdc8_babel/tree/main/ug_example}{https://github.com/cronologic-de/xhptdc8\tu babel}. 
		The repository contains a complete project for Microsoft Visual Studio that you can use to compile the example.
		Examples for more programming languages such as Golang, Rust, Python and LabView will be added to the repository over time.

		At the time this document is written, the following source code can be found in the repository:
		\begin{itemize}
			\item \href{https://github.com/cronologic-de/xhptdc8_babel/tree/main/dummy}{dummy library} to be able to develop code for the \deviceName\ without a physical board beeing present
			\item \href{https://github.com/cronologic-de/xhptdc8_babel/tree/main/util}{utility library} to configure the device from YAML strings or YAML files 
			\item command line info tool to list information about all \deviceName\ boards in the system. This tool is written in Go.
		\end{itemize}

		\lstinputlisting{xhptdc/example.cpp}
	}
	% there were problems with underscores in file names so we removed them

