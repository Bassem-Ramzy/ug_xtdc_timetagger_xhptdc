For each measured edge the \deviceName\ creates a 12 byte data structure TDCHit that contains a 64 bit timestamp in picoseconds and three fields with additional information. 

\section{Structure TDCHit}
\label{TDCHit}
\cronvar{long long}{time}\\
The time stamp of the hit in picoseconds. If grouping is enabled the timestamp is relative to the trigger or the separate zero reference of the group. 
Otherwise the timestampe is continuously counting up from the call to \textsf{start\tu capture()}.

\cronvar{unsigned char}{channel}\\
For the forst in the system this 0 to 7 for the TDC channels A to H. 8 or 9 for ADC data. Data from channels 8 and 9 should usually be treated as data from the same channel. 
For the other boards the channel number is incremented by $board\_id \cdot 10$

\cronvar{unsigned char}{type}\\
Additional information on the type of hit recorded. See section \ref{hittypes}.

\cronvar{unsigned short}{bin}\\
For ADC hits this contains the sampled voltage. For TDC hits the content is undefined.

\newpage
\subsection{TDCHit Types \label{hittypes}}
\newcommand{\HTYPE}{\PREFIX TDCHIT\tu TYPE\tu}

\paragraph*{Type information for TDC measurements}
If the hit is a TDC measurement on channels A to H the following flags are defined for the \textsf{type} field of the TDCHit structure:\\*
\begin{tabular}{lc}
    \crondef{\HTYPE RISING} & 0x01\\
    \indent Rising edge &\\
    \crondef{\HTYPE ERROR}  & 0x02\\
    \indent any type of error& \\
    \crondef{\HTYPE ERROR\tu TIMESTAMP\tu LOST}  & 0x04\\
    \indent Hits missing due to L1 FIFO overflow&\\
    \crondef{\HTYPE ERROR\tu ROLLOVER\tu LOST}  & 0x08\\
    \indent Invalid timestamp due to internal error&\\
    \crondef{\HTYPE ERROR\tu PACKET\tu LOST}  & 0x10\\
    \indent Hits missing due to a lost DMA packet&\\
    \crondef{\HTYPE ERROR\tu SHORTENED}  & 0x20\\
    \indent Hits missing due to a shortened DMA packet&\\
    \crondef{\HTYPE ERROR\tu DMA\tu FIFO\tu FULL}  & 0x40\\
    \indent Hits missing due to L2 FIFO overflow&\\
    \crondef{\HTYPE ERROR\tu HOST\tu BUFFER\tu FULL}  & 0x80\\
    \indent Hits missing due to host buffer overflow &\\
\end{tabular}

If hits are missing the error flag is set on the next hit from the same board that is read out.

\paragraph*{Type information for ADC measurements}
If the hit is an ADC measurement on channels 8 or 9 the following flags are defined for the \textsf{type} field of the TDCHit structure:\\*
\begin{tabular}{lc}
    \crondef{\HTYPE ADC\tu INTERNAL} & 0x01\\
    \indent ADC measurement triggered by TiGer &\\
    \crondef{\HTYPE ADC\tu ERROR}  & 0x02\\
    \indent any type of error& \\
    \crondef{\HTYPE ADC\tu ERROR\tu INVALID\tu TRIGGER}  & 0x08\\
    \indent TRG input violated timing requirements. Data may be corrupted&\\
    \crondef{\HTYPE ADC\tu ERROR\tu DATA\tu LOST}  & 0x10\\
    \indent ADC measurement missing due to overflow of any buffer&\\
\end{tabular}

If hits are missing the error flag is set on the next hit from the same board that is read out.


